\documentclass[a4paper]{article}
\usepackage[utf8]{inputenc}
%\usepackage[margin=2cm,includehead,includefoot]{geometry}
\usepackage[english,french]{babel}
    \newcommand{\en}{\selectlanguage{english}}
    \newcommand{\fr}{\selectlanguage{french}}

\usepackage[pdftex]{graphicx}
\usepackage[usenames]{color}
\usepackage[french]{varioref}
\usepackage{longtable}
\usepackage{gitinfo}
\usepackage[backend=bibtex8,backref=true]{biblatex}
\addbibresource{methodologies-administration.bib}
   \defbibcheck{online}{\iffieldundef{url}{\skipentry}{}}
   \defbibcheck{notonline}{\iffieldundef{url}{}{\skipentry}}
\usepackage[%
	pdftex,%
	bookmarks=false,%
	bookmarksopen=false,%
	bookmarksnumbered=false,%
	pdfborder=0,%
	pdfmenubar=true,%
	pdftoolbar=true,%
	pdfpagelabels=true,%
	pdftitle={Methodologies administratives},%
	pdfauthor={Gr\'egory DAVID},%
	pdfsubject={Méthodologies de travail dans l'administration de l'AMMD},%
	pdfpagelayout={OneColumn},%
	plainpages=true,%
	pdfstartview=FitH%
]{hyperref}
\usepackage{listings}
\lstset{%
  basicstyle=\scriptsize\ttfamily,%
  numbers=left,%
  frame=trBL,%
  breaklines=true,%
  breakatwhitespace=true,%
  escapeinside={(*@}{@*)},%
  extendedchars=false,%
}

\newcommand{\AMMD}{Amicale du Mekanik Metam Disco}
\newcommand{\image}[5][\textwidth]{%
% #1 = largeur de la minipage
% #2 = nom de fichier de l'image a incruster
% #3 = includegraphics options
% #4 = le texte du caption
% #5 = label de l'element
    \begin{figure}
        \centering
            \begin{minipage}[c]{#1}
                \centering
                \includegraphics[#3]{#2}
                \caption{#4}
                \label{#5}
            \end{minipage}
    \end{figure}
}
\newcommand{\tableau}[4]{%
% #1 = parametres des colonnes
% #2 = contenu du tableau
% #3 = le texte du caption
% #4 = label de l'element
    \begin{table}
        \centering
        \begin{tabular}{#1}
            #2
        \end{tabular}
        \caption{#3}
        \label{#4}
    \end{table}
    %\afterpage{\clearpage}
}

\title{Méthodologies et structures fonctionnelles administratives}
\author{\href{mailto:tresorier@ammd.net}{Grégory \bsc{David}}}
\date{\today\\ {\scriptsize r\'evision \gitAbbrevHash{} du \gitAuthorIsoDate}}

\listfiles{}

\begin{document}
% \rhead{}\chead{}\lhead{}
% \headrule
% \lfoot{}\cfoot{}\rfoot{\thepage}
% \footrule

\maketitle
\tableofcontents{}
\listoftables{}
\listoffigures{}
\newpage
\section{Administratif}
\label{sec:Tresorerie}

\subsection{Faire un contrat de cession}
\label{sec:FaireContratCession}
Utiliser le modèle \LaTeX{} pour la création des contrats de cession Libre de l'AMMD à l'adresse : \url{https://github.com/AMMD/contrat_de_cession} ou en clone le dépôt : \lstinline{git clone git@github.com:AMMD/contrat_de_cession.git}.


\subsubsection{Création d'un nouveau contrat}
\label{sec:CreationNouveauContratCession}

\paragraph{Nouveau contrat vierge}
\label{sec:NouveauContratCessionVierge}
\begin{enumerate}
    \item création de la copie local du dépôt git :
    \lstinline{git clone https://github.com/AMMD/contrat_de_cession.git}
    \item on se déplace dans le nouveau répertoire créé :
    \lstinline{cd contrat_de_cession}
    \item création de la nouvelle branche de travail :
    \lstinline{git checkout -b nomDuGroupe-lieu-date-WIP}

    \item réaliser les modifications dans le fichier
    \lstinline{config.tex}
    \item modifier les articles du contrat dans le fichier
    \lstinline{main.tex} (si besoin, et en discussion avec
    l'organisateur)
    \item ajouter les fichiers modifiés à la liste de la prochaine
    validation git : \lstinline{git add config.tex main.tex}

    \item valider les modifications faites dans git :
    \lstinline{git commit -m "Message de validation"}
    \item pousser les validation sur le dépôt central :
    \lstinline{git push origin nomDuGroupe-lieu-date-WIP}
    \item refaire les points 4, 5, 6, 7 et 8 tant que le contrat
    n'est pas validé par les deux partenaires (tourneur et
    organisateur)
    \item lorsque le contrat est validé par les deux partenaires :
    \lstinline{git tag nomDuGroupe-lieu-date} (on peut faire
      une signature numérique GPG du tag en ajoutant l'option
      \lstinline{-s})
    \item on pousse sur le dépôt central le tag nouvellement créé
    : \lstinline{git push nomDuGroupe-lieu-date}
    \item puis supprimer la branche sur laquelle nous travaillions
    : \lstinline{git branch -D nomDuGroupe-lieu-date-WIP}
    \item on pousse sur le dépôt central la suppression de branche
    : \lstinline{git push origin --delete nomDuGroupe-lieu-date-WIP}
\end{enumerate}

\paragraph{Nouveau contrat depuis un existant}
\label{sec:NouveauContratCessionDepuisExistant}
\begin{enumerate}
    \item \lstinline{git clone https://github.com/AMMD/contrat_de_cession.git}
    \item \lstinline{cd contrat_de_cession}
    \item on change de branche pour aller dans celle qui contient déjà
    des informations : \lstinline{git checkout branche-ou-tag-existant}
    \item création de la nouvelle branche de travail :
    \lstinline{git checkout -b nomDuGroupe-lieu-date-WIP}
    \item réaliser les modifications dans le fichier
    \lstinline{config.tex}
    \item modifier les articles du contrat dans le fichier
    \lstinline{main.tex} (si besoin, et en discussion avec
    l'organisateur)
    \item ajouter les fichiers modifiés à la liste de la prochaine
    validation git : \lstinline{git add config.tex main.tex}
    \item valider les modifications faites dans git :
    \lstinline{git commit -m "Message de validation"}
    \item pousser les validation sur le dépôt central :
    \lstinline{git push origin nomDuGroupe-lieu-date-WIP}
    \item refaire les points 4, 5, 6, 7 et 8 tant que le contrat n'est
    pas validé par les deux partenaires (tourneur et organisateur)
    \item lorsque le contrat est validé par les deux partenaires :
    \lstinline{git tag nomDuGroupe-lieu-date} (on peut faire
      une signature numérique GPG du tag en ajoutant l'option
      \lstinline{-s})
    \item on pousse sur le dépôt central le tag nouvellement créé :
    \lstinline{git push nomDuGroupe-lieu-date}
    \item puis supprimer la branche sur laquelle nous travaillions :
    \lstinline{git branch -D nomDuGroupe-lieu-date-WIP}
    \item on pousse sur le dépôt central la suppression de branche :
    \lstinline{git push origin --delete nomDuGroupe-lieu-date-WIP}
\end{enumerate}

\subsubsection{Modification d'un contrat de cession}
\label{sec:ModificationContratCession}
\begin{enumerate}
    \item éditer et modifier le fichier \lstinline{config.tex}
    en accord avec les informations en votre possession
    \item vérifier que toutes les variables sont adaptées et qu'il ne
    reste pas de \lstinline{CHANGE-ME}
    \item compiler le document
\end{enumerate}

\subsubsection{Compilation d'un contrat de cession}
\label{sec:CompilationContratCession}
\begin{enumerate}
    \item première compilation LaTeX : \lstinline{pdflatex main}
    \item deuxième compilation LaTeX pour avoir les références
    internes correctes : \lstinline{pdflatex main}
    \item affichage du contrat de cession :
    \lstinline{evince main.pdf}
\end{enumerate}

\subsubsection{Ajout du contrat de cession validé à la facture dans le Dolibarr}
\label{sec:DolibarrContratCession}
\begin{enumerate}
    \item récupérer la version du contrat de cession adaptée
    (Cf. \vref{sec:NouveauContratCessionDepuisExistant})
    \item compiler le contrat de cession (Cf. \vref{sec:CompilationContratCession})
    \item renommer le fichier de contrat de cession :
    \lstinline{mv main.pdf nomDuGroup-lieu-date.pdf}
    \item charger en tant que \emph{fichier joint} le fichier
    \lstinline{nomDuGroupe-lieu-date.pdf} à la facture client (Cf. \vref{sec:DolibarrFactureClient})
\end{enumerate}

\section[Dolibarr, logiciel de gestion]{Dolibarr}
\label{sec:Dolibarr}
Dolibarr ERP \& CRM est un logiciel modulaire (on n'active que les
fonctions que l'on désire) de gestions de TPE/PME, d'indépendants,
d'entrepreneurs ou d'associations.

\subsection{Gestion de projet}
\label{sec:DolibarrGestionProjet}

\subsubsection{Ajouter un projet}
\label{sec:DolibarrAjouterProjet}

\subsubsection{Ajouter des tâches}
\label{sec:DolibarrAjouterTaches}

\subsubsection{Export des calandriers de projet}
\label{sec:DolibarrExportCalendrierProjet}


\subsection{Ajouter un nouveau contact}
\label{sec:DolibarrAjouterNouveauContact}

\subsection{Ajouter un nouveau client}
\label{sec:DolibarrAjouterNouveauClient}

\subsection{Ajouter un nouveau fournisseur}
\label{sec:DolibarrAjouterNouveauFournisseur}

\subsection{Créer une facture client}
\label{sec:DolibarrFactureClient}

\subsection{Créer une facture fournisseur}
\label{sec:DolibarrFactureFournisseur}

\subsection{Ajouter un nouveau service à vendre ou à acheter}
\label{sec:DolibarrNouveauService}

\subsection{Ajouter un nouveau produit à vendre ou à acheter}
\label{sec:DolibarrNouveauProduit}

\subsection{Faire un rapprochement bancaire}
\label{sec:DolibarrBanqueRapprochement}

\subsection{Créer un nouveau bon de commande fournisseur}
\label{sec:DolibarrCommandeFournisseur}

\subsection{Créer un nouveau bon de commande client}
\label{sec:DolibarrCommandeClient}

\subsection{Créer un devis, une proposition commerciale}
\label{sec:DolibarrDevisPropal}

\subsection{Créer un contrat de services}
\label{sec:DolibarrContratServices}

\subsection{Enregistrer le paiement d'une facture fournisseur}
\label{sec:DolibarrPaiementFactureFournisseur}

\subsection{Enregistrer le règlement d'une facture client}
\label{sec:DolibarrPaiementFactureClient}

\subsection{Expédier une commande}
\label{sec:DolibarrExpedierCommande}

\subsection{Gestion des adhérents}
\label{sec:DolibarrGestionAdherents}

Agenda partagé
Point de vente/Caisse enregistreuse
Réalisation de sondages
EMailing de masses vers les clients, prospect ou utilisateurs Dolibarr
Suivi des marges
Recolte de dons 


\section{Démarches, Recettes et HowTo}
\label{sec:HowTo}

\subsection{Je veux vendre mes CD, DVD, Tee-Shirt, etc.}
\label{sec:HowToVendreMerchandising}

\subsection{Je veux pouvoir vendre mon spectacle}
\label{sec:HowToVendreSpectacle}
Ajouter un nouveau service en vente
Ajouter les services humains associés en achat

\subsection{J'ai été contacté pour faire une représentation}
\label{sec:HowToFaireRepresentation}
Renseigner le Tiers comme client

\subsection{Je veux démarcher un lieu de représentation}
\label{sec:HowToDemarcherUnLieu}
Renseigner le Tiers comme prospect
Contacter le lieu et renseigner l'information
Envoyer une proposition commerciale

\subsection{Je veux faire intervenir des musiciens/techniciens lors d'une représentation}
\label{sec:HowToInterventionFournisseur}
Faire budget (Cf. tableau budget équilibré)


\subsection{L'organisateur a accepté mon devis}
\label{sec:HowToAcceptationDevis}

\subsection{Je veux intégrer le calendrier d'un de mes projets dans mon client mail (Thunderbird, Gmail, etc.)}
\label{sec:HowToIntegrerCalendar}




% References bibliographiques
\newpage \printbibheading
\printbibliography[nottype=online,check=notonline,heading=subbibliography,title={Bibliographiques}]
\printbibliography[check=online,heading=subbibliography,title={Webographiques}]
\nocite{*}
\end{document}
